\documentclass{document}

%Packages
\usepackage[utf8]{inputenc}
\usepackage{color}

\begin{document}

        \title{Inteligência Artificial no Processo de Seleção de Candidatos a Vagas de Emprego}
        \author{Francisca Amanda Miranda de Paula - Matrícula: 1812082052}
        \author{Matheus Henrique Santos Lucas - Matrícula: 1822082032}

    \capter{Resumo}

    O presente projeto consiste em desenvolver um aplicativo utilizando a linguagem flutter e o artifício da inteligência artificial para 
    selecionar automaticamente o candidato que se encaixa melhor com a oportunidade de emprego. Com o objetivo de ter um projeto mais limpo, bem
    implementado e organizado, o desenvolvimento do aplicativo levou em consideração alguns critérios, como o uso da arquitetura limpa (Clean 
    Architecture), o padrão de projeto MVVM, além do uso do framework spring boot na parte de API, que foi desenvolvida na linguagem Java. 

    \capter{Abstract}
    
    The project consists of developing an application using flutter language and realizing artificial intelligence to
    automatically select the candidate who best fits the job opportunity. In order to have a cleaner design, as well as
    implemented and organized, the development of the application took into account some criteria, such as the use of clean architecture (Clean
    Architecture), the MVVM design pattern, in addition to the use of the spring boot framework in the API part, which was developed in the Java 
    language.

    \section{Introdução}

    É possível perceber que o uso da inteligência artificial dentro da área de recursos humanos traz muitos benefícios ao agilizar os processos
    de contratação, além de torná-los mais eficazes. Sendo assim, para o projeto proposto, serão estudadas algumas técnicas de inteligência 
    artificial de modo a conseguir a pré-seleção de candidatos às vagas de forma automática.
    
    \section{Contextualização}

    Tendo em vista que a procura de emprego é algo bastante recorrente, o processo de contratação pode se tornar lento na maioria das vezes pelo
    falo da grande quantidade de perfis a serem analisados, além de ter que levar em consideração os requisitos da vaga. Sendo assim, o trabalho 
    tem a finalidade de agilizar os processos de contratação do RH.  
    
    \section{Problema}

    Definir a melhor estratégia ou técnica da inteligência artificial para usá-la na pré-seleção de funcionários em uma vaga específica, de
    modo a ter os melhores resultados e de forma menos complexa. A partir disso, serão definidos os parâmetros necessários para fazer a tal 
    seleção. Dentro da área de IA, existem inúmeras técnicas convenientes que auxiliam na tomada de decisão, uma delas, é conhecido como 
    sistema especialista, que como o próprio nome diz, faz o trabalho de um especialista em uma determinada área de conhecimento. A partir disso,
    seria viável utilizar um sistema especialista para a fazer o trabalho do especialista de RH, para fazer a seleção inicial de 
    candidatos? E como definir o melhor candidato, tendo em vista que a quantidade de parãmetros para medir as qualidades de um profissional são
    praticamente infinitas?

    \section{Objetivo Geral}

    Desenvolver uma aplicação para dispositivos móveis com sistema operacional Android que facilite a seleção de candidatos às vagas de emprego,
    utilizando para este fim, a inteligência artificial. 

    \section{Objetivo Específico}

    ->Desenvolvimento e consumo de APIs; 
    ->Uso de padrões de projeto e boas práticas de programação, como os princípios S.O.L.I.D e Clean Architecture; 
    ->Planejamento e desenvolvimento da interface do usuário; 
    ->Integração do Backend com a Interface do usuário; 
    ->Desenvolvimento de funcionalidades como cadastro e autenticação de usuários;
    ->Aplicar a inteligência artificial para a tomada de decisão.

    \section{Referencial Teórico}

    É possível perceber que a inteligência artificial está cada vez mais presente dentro das empresas, porém, o seu processo de aplicação em uma 
    determinada área pode ser complexo e lento à depender da forma como é feito a coleta de dados e o treinamento da IA. Uma das técnicas 
    utilizadas pela IA, se chama sistema especialista que consiste em um sistema que se comporta como um ser humano experiente em uma área
    específica, sendo esta definição, coerente com a finalidade do trabalho proposto. Na aplicação envolvendo a área de recursos humanos, 
    será necessário coletar dados a partir de um profissional experiente na sua área de atuação para que possa transmitir os seus conhecimentos
    para o sistema especialista, para a criação da sua base de conhecimento ou regras, que serão utilizadas para a tomada de decisão, ou seja,
    selecionar ou não o candidato à vaga. 

    Tendo em vista que o processo de contratação não é tão simples, pois existem muitos pontos que o recrutador utiliza como avaliação, pois
    não são somente os conhecimentos que estão sendo avaliados em alguns casos, mas também levam em conta o comportamento, as atitudes e o
    aspecto visual, podendo haver casos de erro ao selecionar um candidato. Por outro lado, nem sempre só o conhecimento importa, pois, dependendo
    do perfil da empresa e da área de trabalho, é importante levar em consideração as qualidades e defeitos, além da personalidade do interessado à vaga.

    \section{Trabalhos Correlatos}

    Entre inúmeras tecnologias existentes que automatizam processos, existem as que auxiliam no processo de contratação de candidatos às vagas
    de emprego. Como exemplo, tem-se a ferramenta HireVue, utilizada em entrevistas online, que é capaz de interpretar a análise da linguagem
    corporal, expressões faciais e tom de voz. Como estratégia de avaliação para decidir se o candidato é bom ou ruim, o entrevistado é comparado
    com os melhores funcionários da empresa, e a partir disso, selecionam os melhores para os recrutadores.

    Além da técnica de análise comportamental, mencionada anteriormente, também é possível utilizar algoritmos para identificar o perfil das pessoas
    nas redes sociais, podendo ser utilizado para fazer uma pré-seleção, diminuindo a carga horária de tranbalho do recrutador. Por outro lado,
    a inteligência artifical pode ser ainda mais útil, pois além de poder armazenar e manipular dados, ela também é capaz de adquirir, 
    representar e manipular novos conhecimentos e, a partir disso, ela consegue fazer a inferência e dedução de novos conhecimentos, podendo 
    utilizar estes métodos para resolver problemas e aperfeiçoar os processos do RH.

    As técnicas de inteligência artifical que podem ser utilizadas para selecionar interessados em uma vaga de emprego, são: Lógica
    Fuzzy, Sistema Especialista, Redes Neuronais Artificiais, Sistemas Baseados em Casos e Algoritmos Genéricos. O uso de Redes Neuronais Artificiais
    (RNA) auxilia pelo motivo de ter a capacidade de prever os candidatos que seriam os mais adequados às vagas e tem uma enorme habilidade
    em aprender no ambiente em que está inserido. Assim, o RNA, com o aumento das informações que adquire ao longo das avaliações dos currículos dos
    candidatos, cada vez mais a sua escolha se tornará mais assertiva, com base na experiência.

    \section{Resultados Esperados}

    O aplicativo de Inteligência Artificial no Processo de Seleção de Candidatos a Vagas de Emprego, deverá ser capaz de substituir algumas 
    tarefas que cabem ao especialista de recursos humanos dentro de uma empresa. Assim, a aplicação ficará responsável por fazer
    o trabalho da pré-seleção de candidatos, de modo que o selecionado para a próxima etapa do processo seletivo, seja adequado para o perfil da 
    vaga, facilitando assim, o trabalho na hora da contratação, ao agilizar o processo de recrutamento, visto que é comum ter uma grande 
    quantidade de candidatos à uma determinada vaga de emprego. Portanto, a aplicação deverá ter consigo, todas as funcionalidades básicas de
    uma API, funcionando corretamente, como o cadastro e autenticação de usuários, manipulação e alteração de dados com o auxílio do banco de
    dados do tipo relacional etc. Além disso, é esperado que o sistema seja fácil de ser compreendido e de fazer as manutenções necessárias sem
    dificuldade, e para isto, houve uma grande preocupação em se utilizar uma arquitetura de projeto limpa (Clean Architecture) e boas práticas 
    de programação com o auxílio do padrão de projeto MVVM e a aplicação dos princípios S.O.L.I.D.

    \section{Cronograma Completo}

    23/03 - 30/03:
        ->Criação do repositório no github;
        ->Levantamento de requisitos;
        ->Diagramação de classes;
        ->Definição da arquitetura e do padrão de projeto a ser utilizado;
    31/03 - 09/04:
        ->Desenvolvimento das APIs;
        ->Desenvolvimento da primeira parte do artigo;
        ->Desenvolvimento da interface do usuário;
        ->Criação das telas (onboarding, login, alteração de senha e cadastro de usuário);
    10/04 - 27/04:
        ->Gestão de dependências; 
    28/04 - 04/05:
        ->Criação das telas específicas do tema do projeto;
    05/05 - 11/05: 
        ->Consumo das APIs e integração com a interface do usuário;
        ->Definição da estrutura de relacionamento das tabelas no banco de dados PostgreSQL;
    12/05 - 18/05:
        ->Desenvolvimento das funcionalidades de cadastro e autenticação de usuários; 
    19/05 - 08/06:
        ->Uso da inteligência artificial no aplicativo;
        ->Desenvolvimento da segunda parte do artigo.
    
    \section{Bibliografia}
    \begin{thebibliografia}{00}
    \bibitem{b1} F. Bo-Shone, "Technology, Jobs e the Future of Work in Australia", Julho 2021.
    \bibitem{b2} M. Afonso Paulo, R. Brenno Anderson, A. Cristine Amora e V. Rosângela Couras, "INTELIGÊNCIA ARTIFICIAL - RECURSOS HUMANOS FRENTE AS NOVAS TECNOLOGIAS, POSTURAS E ATRIBUIÇÕES", Outubro 2018.
    \bibitem{b3} T. Ivana, "The Attitude of Job Candidates towards Artificial Intelligence in Hiring Process", Maio 2020.
    \bibitem{b4} S. Abhilasha e S. Apurva, "Impact of Artificial Intelligence on HR practices in the UAE".
    \bibitem{b4} R. Geetha e D. Bhanu Sree, "RECRUITMENT THROUGH ARTIFICIAL INTELLIGENCE: A CONCEPTUAL STUDY ".
    \bibitem{b4} J. Mariana Namen, "INTELIGÊNCIA ARTIFICIAL NO RECRUTAMENTO & SELEÇÃO:INOVAÇÃO E SEUS IMPACTOS PARA A GESTÃO DE RECURSOSHUMANOS", Fevereiro 2020.
    \end{thebibliography}
\end{document}